\documentclass[12pt]{article}
\usepackage{url,graphicx,tabularx,array}

%%%%%%%%%%%%%%%%%%%%%%%%%%%%%%%%%%%%%%%%%%%%%%%%%%
%%%%%%%%%%%%%%%%%%%% PREAMBLE %%%%%%%%%%%%%%%%%%%%
%%%%%%%%%%%%%%%%%%%%%%%%%%%%%%%%%%%%%%%%%%%%%%%%%%

% -------------------- defaults -------------------- %
% load lots o' packages

% layout control
\usepackage[paper=a4paper,left=25mm,right=25mm,top=20mm,bottom=25mm]{geometry}
\usepackage[onehalfspacing]{setspace}
\setlength{\parskip}{.5em}
\usepackage{rotating}
\usepackage{setspace}

% math typesetting
\usepackage{array}
\usepackage{amsmath}
\usepackage{amssymb}
\usepackage{amsfonts}

\usepackage{verbatim}
\usepackage{listings}

% restricts float objects to be inserted before end of section
% creates float barriers
\usepackage[section]{placeins}

% tables
\usepackage{tabularx}
\usepackage{booktabs}
\usepackage{multicol}
\usepackage{multirow}
\usepackage{longtable}

\usepackage[%
decimalsymbol=.,
digitsep=fullstop
]{siunitx}

% to adapt caption style
\usepackage[font={small},labelfont=bf]{caption}

% references
\usepackage[longnamesfirst]{natbib}
\bibpunct{(}{)}{;}{a}{}{,}

% footnotes at bottom
\usepackage[bottom]{footmisc}

% to change enumeration symbols begin{enumerate}[(a)]
\usepackage{enumerate}

% to make enumerations and itemizations within paragraphs or
% lines. f.i. begin{inparaenum} for (a) is (b) and (c)
\usepackage{paralist}

% to colorize links in document. See color specification below
\usepackage[pdftex,hyperref,x11names]{xcolor}

% for multiple references and insertion of the word "figure" or "table"
% \usepackage{cleveref}

% load the hyper-references package and set document info
\usepackage[pdftex]{hyperref}

% graphics stuff
\usepackage{coordsys}
\usepackage{tikz}
\usepackage{subfig}
\usepackage{graphicx}
\usepackage[space]{grffile} % allows us to specify directories that have spaces
\usepackage{placeins} % prevents floats from moving past a \FloatBarrier
%\usepackage{tikz}
% \usepackage{pgfplots}

% define clickable links and their colors
\hypersetup{%
unicode=false,          % non-Latin characters in Acrobat's bookmarks
pdftoolbar=true,        % show Acrobat's toolbar?
pdfmenubar=true,        % show Acrobat's menu?
pdffitwindow=false,     % window fit to page when opened
pdfstartview={FitH},    % fits the width of the page to the window
pdfnewwindow=true,%
pagebackref=false,%
pdfauthor={Matt Dickenson},%
pdftitle={Title},%
colorlinks,%
citecolor=black,%
filecolor=black,%
linkcolor=black,%
urlcolor=RoyalBlue4}

\lstset{
    basicstyle=\ttfamily,
    upquote=true,
    breaklines=true,
    tabsize=2,
    postbreak=\raisebox{0ex}[0ex][0ex]{\ensuremath{\hookrightarrow}},
    breakatwhitespace=true,
    numbers=left,
    showstringspaces=false
}

% -------------------------------------------------- %



%-- Set options
\setlength{\parskip}{1ex} %--skip lines between paragraphs
\setlength{\parindent}{0pt} %--don't indent paragraphs
\newcommand{\inputTikZ}[2]{%  
     \scalebox{#1}{\input{#2}}  
}


%-- Commands for header
\renewcommand{\title}[1]{\textbf{#1}\\}
\renewcommand{\line}{\begin{tabularx}{\textwidth}{X>{\raggedleft}X}\hline\\\end{tabularx}\\[-0.5cm]}
\newcommand{\leftright}[2]{\begin{tabularx}{\textwidth}{X>{\raggedleft}X}#1%
& #2\\\end{tabularx}\\[-0.5cm]}
%\linespread{2} %-- Uncomment for Double Space
\begin{document}

% ----------------------TITLE----------------------- %
\title{CS 201 - HW 6 Analysis}
\line
\leftright{\today}{Matt Dickenson} %-- left and right positions in the header
\setlength{\parindent}{16pt}

For the benchmarking and empirical analysis described below I ran the code in \texttt{BoggleStats.java} with various combinations of Lexicons and Autoplayers, and at several values of \texttt{NUM\_TRIALS}. All trials were conducted on my 2011 Macbook Pro with an Intel Dual Core processor, with minimal additional applications running in the background. The seed for all trials was set to 12345 to ensure comparability between results. Minimum word length for all trials was set at 3 letters. 

\section*{Part A: High Scores}

The autoplayers' high scores for different numbers of simulations are shown in Table \ref{highscores}. The more trials are run, the larger the proportion of potential board layouts that is explored. As the table shows, however, even for some large increases in the search space there is no increase in the maximum score. High scores shown are the same for both the \texttt{LexiconFirst} and \texttt{BoardFirst} autoplayers. In all cases the high score is larger with a $5\times5$ board than with a $4 \times 4$ board because more cominations of letters and longer words are possible.

\begin{table}[ht]
\begin{center}
\caption{High Scores for Autoplayers}
\label{highscores}
\begin{tabular}{ccc}
Number of iterations & High score ($4 \times 4$) & High score ($5 \times 5$) \\
\hline
1,000 & 889 & 1,301 \\
10,000 & 889 & 2,120 \\
50,000 & 1,011 & 2,120 \\
\hline
\end{tabular}
\end{center}
\end{table}

\section*{Part B: Running Times}



\end{document}
