\documentclass[12pt]{article}
\usepackage{url,graphicx,tabularx,array}

%%%%%%%%%%%%%%%%%%%%%%%%%%%%%%%%%%%%%%%%%%%%%%%%%%
%%%%%%%%%%%%%%%%%%%% PREAMBLE %%%%%%%%%%%%%%%%%%%%
%%%%%%%%%%%%%%%%%%%%%%%%%%%%%%%%%%%%%%%%%%%%%%%%%%

% -------------------- defaults -------------------- %
% load lots o' packages

% layout control
\usepackage[paper=a4paper,left=25mm,right=25mm,top=20mm,bottom=25mm]{geometry}
\usepackage[onehalfspacing]{setspace}
\setlength{\parskip}{.5em}
\usepackage{rotating}
\usepackage{setspace}

% math typesetting
\usepackage{array}
\usepackage{amsmath}
\usepackage{amssymb}
\usepackage{amsfonts}

\usepackage{verbatim}
\usepackage{listings}

% restricts float objects to be inserted before end of section
% creates float barriers
\usepackage[section]{placeins}

% tables
\usepackage{tabularx}
\usepackage{booktabs}
\usepackage{multicol}
\usepackage{multirow}
\usepackage{longtable}

\usepackage[%
decimalsymbol=.,
digitsep=fullstop
]{siunitx}

% to adapt caption style
\usepackage[font={small},labelfont=bf]{caption}

% references
\usepackage[longnamesfirst]{natbib}
\bibpunct{(}{)}{;}{a}{}{,}

% footnotes at bottom
\usepackage[bottom]{footmisc}

% to change enumeration symbols begin{enumerate}[(a)]
\usepackage{enumerate}

% to make enumerations and itemizations within paragraphs or
% lines. f.i. begin{inparaenum} for (a) is (b) and (c)
\usepackage{paralist}

% to colorize links in document. See color specification below
\usepackage[pdftex,hyperref,x11names]{xcolor}

% for multiple references and insertion of the word "figure" or "table"
% \usepackage{cleveref}

% load the hyper-references package and set document info
\usepackage[pdftex]{hyperref}

% graphics stuff
\usepackage{coordsys}
\usepackage{tikz}
\usepackage{subfig}
\usepackage{graphicx}
\usepackage[space]{grffile} % allows us to specify directories that have spaces
\usepackage{placeins} % prevents floats from moving past a \FloatBarrier
%\usepackage{tikz}
% \usepackage{pgfplots}

% define clickable links and their colors
\hypersetup{%
unicode=false,          % non-Latin characters in Acrobat's bookmarks
pdftoolbar=true,        % show Acrobat's toolbar?
pdfmenubar=true,        % show Acrobat's menu?
pdffitwindow=false,     % window fit to page when opened
pdfstartview={FitH},    % fits the width of the page to the window
pdfnewwindow=true,%
pagebackref=false,%
pdfauthor={Matt Dickenson},%
pdftitle={Title},%
colorlinks,%
citecolor=black,%
filecolor=black,%
linkcolor=black,%
urlcolor=RoyalBlue4}

\lstset{
    basicstyle=\ttfamily,
    upquote=true,
    breaklines=true,
    tabsize=2,
    postbreak=\raisebox{0ex}[0ex][0ex]{\ensuremath{\hookrightarrow}},
    breakatwhitespace=true,
    numbers=left,
    showstringspaces=false
}

% -------------------------------------------------- %



%-- Set options
\setlength{\parskip}{1ex} %--skip lines between paragraphs
\setlength{\parindent}{0pt} %--don't indent paragraphs
\newcommand{\inputTikZ}[2]{%  
     \scalebox{#1}{\input{#2}}  
}


%-- Commands for header
\renewcommand{\title}[1]{\textbf{#1}\\}
\renewcommand{\line}{\begin{tabularx}{\textwidth}{X>{\raggedleft}X}\hline\\\end{tabularx}\\[-0.5cm]}
\newcommand{\leftright}[2]{\begin{tabularx}{\textwidth}{X>{\raggedleft}X}#1%
& #2\\\end{tabularx}\\[-0.5cm]}
%\linespread{2} %-- Uncomment for Double Space
\begin{document}

% ----------------------TITLE----------------------- %
\title{CS 201 - HW 3 WriteUp}
\line
\leftright{\today}{Matt Dickenson} %-- left and right positions in the header
\setlength{\parindent}{16pt}

\section*{Part A}

\paragraph{1)} The time to generate order-5 text using \texttt{romeo.txt} is shown in the table below:

\begin{table}[h]
\caption{Run Times for Romeo and Juliet with Brute Force Method}
\begin{center}
\begin{tabular}{lr}
Characters & Time (seconds) \\
\hline
100 & 0.042 \\
200 & 0.083 \\
400 & 0.162 \\
800 & 0.338 \\
1600 & 0.641 \\
\end{tabular}
\end{center}
\end{table}

As the table shows, doubling the size of the requested output approximately doubles the size of the time it takes to generate the text, plus a fixed amount of time of about 2 milliseconds. Thus, the brute force model is $O(n)$where $n$ is the size of the output.  

An order-10 model with this same textual input takes about the same amount of time, but an order-1 model seems to take about 1.4 to 1.5 times as long at all sizes of input. This is likely because the number of 1-grams is much larger than the number of 5- or 10-grams in the text. 

\paragraph{2)} The size of the input affects the run time through the \texttt{while} loop on line 71 of \texttt{MarkovModel.java}. Since the run time of nested loops are multiplicative, the run time is a function of $n \times m$ where $n$ is the length of the requested output and $m$ is the length of the input text. In (A1) we held $m$ fixed. If we change the input text from \emph{Romeo and Juliet} to \emph{Scarlet Letter} (which is over 3 times as long), I would expect the runtime time to increase by $3n$. 

Consequently, I predict that generating order-5 text of lengths 400, 800, and 1600 characters would take about 0.5, 1.0, and 1.8 seconds, respectively. If we were to use the King James Bible text, $m$ would be roughly 9 times as long as it is with \emph{Scarlet Letter} and about 30 times as long as \emph{Romeo and Juliet}. I predict that generating 1600 characters of text with an order-5 model using the King James corpus would take about 16.2 ($9\times1.8$) seconds. 

\paragraph{3)} Using the Map method in \texttt{MapMarkovModel.java}, the run times are as follows for an order-5 model with \texttt{romeo.txt}:

\begin{table}[h]
\caption{Run Times for Romeo and Juliet with HashMap Method}
\begin{center}
\begin{tabular}{lr}
Characters & Time (seconds) \\
\hline
100 & 0.001 \\
200 & 0.001\\
400 & 0.003 \\
800 & 0.008 \\
1600 & 0.014 \\
\end{tabular}
\end{center}
\end{table}

This data indicates to me that the Map method is still $O(n)$ because the run time approximately doubles as $n$ doubles. However, it does reduce the time required by a factor of about 40. This is one weakness of Big-Oh notation: sometimes constant factors make a big difference! 

\section*{Part B}

The table and figure below show how the runtimes of HashMap and TreeMap depend on the size of the training text corpus. In each case the Map is used to generate 2000 words using an order-5 Markov model. The additional texts that I used were the US Constitution and the Declaration of Independence, both obtained from \url{http://www.usconstitution.net}.

\begin{table}[h]
\caption{Run Times for Building Maps with Hash and Tree}
\begin{center}
\begin{tabular}{lrrrr}
Corpus Name & Length (words) & Key Count & HashMap &  TreeMap \\
\hline
alice & 26,468 & 26,222 & 0.008  & 0.009 \\
clinton-nh & 887 & 871 & 0.013 & 0.003  \\
const & 7,652 & 7,049 & 0.004 & 0.005 \\ 
declar & 1,340 & 1,340 & 0.010 & 0.002 \\
edwards-nh & 817 & 800 & 0.002 & 0.001 \\ 
hawthorne & 85,754 & 85,590 & 0.027 & 0.005 \\
melville & 14,353 & 14,316 & 0.021 & 0.002 \\
obama-nh & 1,079 & 1,073 & 0.003 & 0.001 \\
poe & 2,324 & 2,323 & 0.012 & 0.001 \\
romeo & 25,788 & 25,740 & 0.003 & 0.003 
\end{tabular}
\end{center}
\end{table}

\begin{center}
\includegraphics[scale=0.75]{plotB.pdf}
\end{center}

The plot also includes local smoothers to help show the relationship between key count and run time for the two map implementation. As you can see in the table, TreeMap takes only a fraction of the time that the HashMap implementation does. The run time for TreeMap also seems to increase more slowly with respect to the key count than HashMap does. This suggests that the \texttt{compareTo} function (used by TreeMap) is more efficient than \texttt{hashCode} (used by HashMap). For a corpus of text with more than 1000 words, the TreeMap is clearly the preferred implementation. 

\end{document}
